% Created 2017-05-09 ter 15:10
% Intended LaTeX compiler: pdflatex
\documentclass[11pt]{article}
\usepackage[utf8]{inputenc}
\usepackage[T1]{fontenc}
\usepackage{graphicx}
\usepackage{grffile}
\usepackage{longtable}
\usepackage{wrapfig}
\usepackage{rotating}
\usepackage[normalem]{ulem}
\usepackage{amsmath}
\usepackage{textcomp}
\usepackage{amssymb}
\usepackage{capt-of}
\usepackage{hyperref}
\author{Pedro Cunial}
\date{\today}
\title{Relatório do Projeto\\
  \large Longest Common Subsequence}
\hypersetup{
 pdfauthor={Pedro Cunial},
 pdftitle={Relatório do Projeto},
 pdfkeywords={},
 pdfsubject={},
 pdfcreator={Pedro Cunial},
 pdflang={Portuguese}}
\begin{document}

\maketitle
\tableofcontents

\begin{abstract}
  Um problema muito comum na biotecnologia é definir a semelhança entre dois
  genomas, ou seja, o quão parecidos são os DNAs de dois seres vivos, seja para
  um teste de paternidade, ou até mesmo para definir o quão distante
  genéticamente o homem está do macaco. Computacionalmente este cálculo torna-se muito
  extenso devido a grande quantidade de redundâncias, neste artigo, pretendo
  discutir a solução deste problema utilizando programação dinâmica, que apesar
\end{abstract}

\end{document}