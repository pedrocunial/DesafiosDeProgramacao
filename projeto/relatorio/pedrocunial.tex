% Created 2017-05-09 ter 15:10
% Intended LaTeX compiler: pdflatex
\documentclass[11pt]{article}
\usepackage[utf8]{inputenc}
\usepackage[T1]{fontenc}
\usepackage{graphicx}
\usepackage{grffile}
\usepackage{longtable}
\usepackage{wrapfig}
\usepackage{rotating}
\usepackage[normalem]{ulem}
\usepackage{amsmath}
\usepackage{textcomp}
\usepackage{amssymb}
\usepackage{capt-of}
\usepackage{listings}
\usepackage{hyperref}
\author{Pedro Cunial}
\date{\today}
\title{Relatório do Projeto\\
  \large \emph{Longest Common Subsequence}}
\hypersetup{
 pdfauthor={Pedro Cunial},
 pdftitle={Relatório do Projeto},
 pdfkeywords={},
 pdfsubject={},
 pdfcreator={Pedro Cunial},
 pdflang={Portuguese}}
\begin{document}

\maketitle
\tableofcontents

\begin{abstract}
  Um problema muito comum na biotecnologia é definir a semelhança entre dois
  genomas, ou seja, o quão parecidos são os \emph{DNAs} de dois seres vivos, seja para
  um teste de paternidade, ou até mesmo para definir o quão distante
  genéticamente o homem está do macaco. Computacionalmente este cálculo torna-se muito
  extenso devido a grande quantidade de redundâncias, neste artigo, pretende-se
  discutir a possível otimização na solução do problema pelo uso da programação dinâmica.
\end{abstract}

\textbf{Palavras Chave:} Programação Dinâmica, Biotecnologia, Algoritmos,
Recursão, Memoização

\section{Introdução}
Muito comum em problemas computacionais relacionados à biotecnologia, conhecido
como \emph{Longest Common Subsequence} busca encontrar a maior subsequência de
caracteres comuns em duas sequências (sejam elas palavras ou apenas sequências
de caracteres). Por definição, uma subsequência difere de uma substring (ou
subtexto) no sentido de que uma subsequência não precisa representar caracteres
contínuos em uma sequência, da forma que \emph{ADFGE} é uma subsequência de
\emph{AEDGFGE} com \emph{DGADEFEGAE}, por mais que não seja uma substring.

\section{Aplicações}

Como o próprio exemplo citado já sugere, o problema pode ser utilizado para
cruzar genomas humanos com o intuito de descobrir o quão semelhantes são, estudo
utilizado em exames de paternidade, por exemplo. Além disso, expandindo o
conceito do problema, se substitui-se a busca de uma subsequência em duas
sequências pela busca de uma subsequência de palavras em dois livros ou textos,
podendo utilizar o mesmo algoritmo como indicador de plágio em obras.

Este segundo exemplo é o utilizado pela maior parte das ferramentas populares de
controle de versão (como o Git) para a reconciliação de mudanças (\texttt{git
  merge}, \texttt{git pull} etc).

\section{Implementação}

\subsection{Definição}
Para encontrar a maior subsequência em duas sequências, pode-se análisar os
caracteres destas sequências par a par até o seu fim, de forma que quando o
resultado for igual encontramos mais um caractere da subsequência.

Desta forma, uma solução ingênua para o problema seria implementar uma recursão
caractere a caractere cruzando-os, gerando uma árvore como a seguinte:

\begin{lstlisting}
                     lcs("AXYT", "AYZX")
                      /                \
        lcs("AXY", "AYZX")            lcs("AXYT", "AYZ")
         /              \              /               \
lcs("AX", "AYZX") lcs("AXY", "AYZ") lcs("AXY", "AYZ") lcs("AXYT", "AY")
\end{lstlisting}
\emph{fonte: http://www.geeksforgeeks.org/dynamic-programming-set-4-longest-common-subsequence/}

\subsection{Melhorias Possíveis}
Pela própria representação da árvore de chamadas da função, percebe-se que o
cálculo redundante da subsequência de ``AXY'' com ``AYZ'' dos dois lados da
árvore. Dado que o cálculo de cada subsequência já é uma operação muito custosa
por si, o cálculo redundante da mesma torna-se um enorme empecilho para a
viabilidade do algoritmo.

Felizmente, este problema pode ser resolvido pela técnica de memoização,
salvando os cálculos já feitos, de forma que cálculos redundantes sejam
substituidos por apenas uma consulta (de tempo constante) em uma estrutura de
dados salvos.

\section{Didática}

\subsection{Principais Pontos}
Em aspéctos didáticos, os pontos mais relevantes a serem discutidos no algoritmo
são as partes em que a memoização pode ser utilizada, assim como qual estrutura
de dados utilizar para a mesma.

Dada a simplicidade do algoritmo em si, a abordagem de como transcrever o código
de \emph{top-down} para \emph{bottom-up}, além de explorar casos solucionáveis
pelo problema, desenvolvendo uma solução em conjunto com a sala em Python -- a
sugestão da solução ser desenvolvida em Python vem da possível dificuldade  e
falta de familiaridade do público com a sintaxe do C ou C++, além das
simplificações didáticas da linguagem.

\subsection{Principais Dificuldades}
Dado o público alvo da aula, um uso excessivo de termos técnicos, principalmente
associados à recursão pode ser problemático. Além disso, como citado
anteriormente a sintaxe do C ou C++ provavelmente será um aspecto problemático,
o desenvolvimento de um código em uma aula de algoritmo seria de grande serviço
para a sua compreensão.

Por fim, o conceito de complexidade pode ser problemático, no entanto, algumas
simplificações podem ser feitas, como por exemplo simplesmente aceitar que uma
consulta na estrutura utilizada para memoização não tem custo computacional
algum, enquanto mais uma iteração do código teria um custo consideravelmente alto.

\section{Conclusão}
O problema \emph{Longest Common Subsequence} é de grande relevância não só para
o estudo de temas da biotecnologia mas também para o estudo e aprendizado de
algoritmos, principalmente de programação dinâmica.

Apesar de não ser tão completo quanto outros casos estudados pela programação dinâmica, o
\emph{Longest Common Subsequence} é um bastante completo, no sentido que sua
implementação ingênua é bastante simples, de forma que transformá-lo em um
algoritmo de programação dinâmica é um processo bastante linear, permitindo um
terceiro passo que seria o transformar do algoritmo top-down em sua
implementação bottom-up.

\end{document}