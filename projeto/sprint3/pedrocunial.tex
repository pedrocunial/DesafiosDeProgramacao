% Created 2017-05-16 ter 20:53
% Intended LaTeX compiler: pdflatex
\documentclass[11pt]{article}
\usepackage[utf8]{inputenc}
\usepackage[T1]{fontenc}
\usepackage{graphicx}
\usepackage{grffile}
\usepackage{longtable}
\usepackage{wrapfig}
\usepackage{rotating}
\usepackage[normalem]{ulem}
\usepackage{amsmath}
\usepackage{textcomp}
\usepackage{amssymb}
\usepackage{capt-of}
\usepackage{hyperref}
\author{Pedro Cunial}
\date{\today}
\title{Sprint 3}
\hypersetup{
 pdfauthor={Pedro Cunial},
 pdftitle={Sprint 3},
 pdfkeywords={},
 pdfsubject={},
 pdfcreator={Emacs 25.1.1 (Org mode 9.0.5)},
 pdflang={English}}
\begin{document}

\maketitle
\tableofcontents


\section{Esqueleto}
\label{sec:org0dc56ee}
Pretendo formular um vídeo seguindo um modelo similar ao do canal Computerphile
do youtube (\url{https://www.youtube.com/user/Computerphile}), criado por acadêmicos
de britânicos de computação em geral. A ideia é introduzir a origem do problema,
seguido da implementação ingênua, da qual pode-se partir para a implementação
topdown. A partir da implementação topdown pode-se explicar melhor os conceitos de
memoização e como a programação dinâmica otimiza o código. Seguindo, explica-se
utilizações comuns do algoritmo e finalmente a implementação bottomup do mesmo.

Um storyboard do video pode ser visto em: \url{https://docs.google.com/presentation/d/1FgQ14rb9tX8wO0tv0V1yqm0nkr294IhXvj44saqLLC8/edit?usp=sharing}

\section{Sketch}
\label{sec:org1484be5}
Para a aula, pretendo fazer algo um pouco mais lúdico em comparação ao vídeo,
simulando um caso onde alguém fora contratado para resolver um caso onde o governo
japonês busca "O Escolhido", filho de Pai Mei, o qual se encontra dando aula no
Insper. Por saberem que sou estudande de computação na faculdade, eles me contrataram
para que descobrisse quem é este escolhido. Minha primeira ideia era de que o próprio
professor Marcelo Hashimoto seria o "escolhido", no entanto, para ter certeza disso
precisaria fazer um teste de DNA para confirmar a paternidade do professor.

Sob a premissa de tentar descobrir a verdadeira origem do professor, desenvolveria a
questão do problema da maior subsequência, onde desejo fazer um teste de paternidade
entre o professor e o Pai Mei (cujo DNA caiu em domínio público).

Dai em diante temos um desenvolvimento análogo ao do vídeo, começando por uma melhor
caracterização do problema, seguido da sua implementação ingênua, a qual pretendo fazer
em pseudo-código na lousa (acredito que uma dinâmica em alguma línguagem de programação
possa acabar atrasando o desenvolvimento da aula por questões menores como sintaxe, por
exemplo). Em seguida, explico o conceito de memoização e adiciono ela ao código,
exemplicando como isso melhora muito o desempenho do mesmo (uma boa forma de mostrar isso
é abrir a "árvore" da recursão e mostrar os cálculos redundantes).

Com isso, mostro que rodei o código e descobri que o verdadeiro "escolhido" era na
verdade o prof. Luciano Soares e finalizo a aula assim.

Não acredito que a implementação bottom up va caber no tempo de 15 minutos, mas não
descarto esta possibilidade.
\end{document}