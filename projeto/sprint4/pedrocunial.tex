% Created 2017-05-26 sex 09:09
% Intended LaTeX compiler: pdflatex
\documentclass[11pt]{article}
\usepackage[utf8]{inputenc}
\usepackage[T1]{fontenc}
\usepackage{graphicx}
\usepackage{grffile}
\usepackage{longtable}
\usepackage{wrapfig}
\usepackage{rotating}
\usepackage[normalem]{ulem}
\usepackage{amsmath}
\usepackage{textcomp}
\usepackage{amssymb}
\usepackage{capt-of}
\usepackage{hyperref}
\author{Pedro Cunial}
\date{\today}
\title{Sprint 4}
\hypersetup{
 pdfauthor={Pedro Cunial},
 pdftitle={Sprint 4},
 pdfkeywords={},
 pdfsubject={},
 pdfcreator={Emacs 25.1.1 (Org mode 9.0.5)},
 pdflang={English}}
\begin{document}

\maketitle
\tableofcontents


\section{Roteiro e Estimativa de Tempo}
\label{sec:orgdd9cb11}
Pretendo fazer uma aula mais lúdica, de forma que a necessidade de uso do algoritmo
viria de uma "história".

A "história" seria que o governo japonês está buscando pelo "Escolhido", que seria o
filho do Pai Mei. Por saberem que o "Escolhido" se encontra no Brasil dando aula no
Insper, o governo japonês acabou me contratando para que descobrisse quem de fato é
este "Escolhido.

Inicialmente, minha sugestão é que ele fosse o professor Marcelo Hashimoto, que
capturassemos ele e o entregassemos ao governo nipônico. No entanto, lembrando que
sua matéria ainda não havia terminado e ainda tinha pendências acadêmicas com o mesmo
decidi pesquisar melhor sobre como descobrir com uma maior certeza se o professor de
fato é o "Escolhido". Pretendo gastar no máximo 3 minutos nesta introdução,
preferencialmente me contendo a 2 minutos.

Com o contexto introduzido, a ideia é já sugerir a análise de DNA como o melhor
método para o caso e, como estudante de computação, minha ideia seria implementar
um código que fizesse a mesma. Sob este contexto, explicarei a análise de subsequências
tal como o próprio conceito de subsequência contra o de substring. Para esta parte da
apresentação não pretendo passar de 1 minuto (totalizando 4 minutos no pior caso até
agora).

Daí vem uma primeira implementação, onde, em pseudo-código, na lousa escreverei um
código bastante ingênuo e que não se baseia em programação dinâmica para resolver o
problema. Apesar da minha ideia inicial ser a de fazer um código em Python, acredito
que o setup do telão e erros tolos da minha parte podem comprometer muito da aula,
não valendo a maior "utilidade" do produzido. Enfim, a ideia é narrar conforme escrevo
esta implementação, algo como "Para uma letra 'i' da minha sequência A, a letra 'j' da
sequência B é igual a ela? Se sim (\ldots{})". Isso não deve se estender por muito tempo,
levando no máximo 3 minutos, mas também não muito menos (totalizando 7 minutos até agora
no pior caso).

Com ela, a ideia é mostrar sua desnecessária complexidade e redundância ao simular a
execução do código em uma espécie de teste de mesa, mostrando os cálculos redundantes
nas chamadas recursivas ao fazer o desenho de uma árvore das chamadas e circulando os
valores já calculados e chamados novamente, o que não deve levar mais de 1 minuto
(totalizando 8 minutos até agora).

Com isso, a ideia é sugerir o uso da memoização para a solução deste problema, no
entanto, a estrutura a ser utilizada não é clara, assim como a forma que faremos ela.
Ai entra o material auxíliar, onde entregarei a grupos de 4 ou mais alunos um material
muito semelhante ao utilizado na aula onde você ensinou pilhas, filas, arrays e matrizes,
ou seja, um tubo fechado aberto de só um lado, um tubo fechado aberto dos dois lados, um
tabuleiro quadriculado e uma sequência de quadrados. A ideia é que os alunos decidam a
melhor estrutura para armazenar os valores já calculados dos respectivos 'i's e 'j's das
sequências (sendo "i" e "j" índices de seus carácteres, conceito que eles já devem estar
familiarizados pela parte anterior da aula). Para isso, não gostaria de passar de 3
minutos pensando entre eles (pois a ideia é que os grupos rapidamente reconheçam que
a matriz é a melhor ideia entre as opções) e mais um minuto (talvez um pouco menos até)
para que apresentassem sua conclusão aos outros, totalizando 12 minutos até agora.

Com a discussão da melhor estrutura, é esperado que pelo menos um dos grupos tenha
chegado em uma forma razoável de como utilizar a matriz para armazenar estes dados,
tal que posso gastar os últimos minutos adaptando o pseudo-código feito com a
nova implementação sugerida.

Para fechar a aula, a ideia era dizer que após aplicar o algoritmo, acabei descobrindo
que o "Escolhido" era na verdade o professor Luciano Soares, filho renegado de Pai Mei.
\end{document}